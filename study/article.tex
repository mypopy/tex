\documentclass{book}
\begin{document}
\title{Study \LaTeX}
\author{JunZhao}
\date{\today}
\maketitle

\noindent This is my \emph{first} document prepared in \LaTeX. I typed it on \today.

We have seen that to typeset something in \LaTeX, we type in the
text to be typeset together with some \LaTeX commands.
Words must be separated by spaces (does not matter how many)
and lines maybe broken arbitrarily.

The end of a paragraph is secified by a \emph{blank line}
in the input. In other words, whenever you want to start a new
paragraph, just leave a blank line and proceed.

Carrots are good for you eyes, since they contain Vitamin A. Have you ever seen a rabibit
wearing glassed?

Carrots are good for you eyes, since they contain Vitamin A\@. Have you ever seen a rabibit
wearing glassed?

The number 1, 2, 3, eth.\ are called natural number. According to Kronecker, they were made
by Cod; all else being the works of Man.

Note the difference i right and left quotes in `signal quotes' and ``double quotes''.

Note the difference i right and left quotes in \lq signal quotes\rq\ and \lq\lq double quotes\rq\rq.

X-rays are discussed in pages 221--225 of Volume 3---the volume on electromagnetic waves.

Maybe I have now learnt about 1\% of \LaTeX.

\begin{center}
The \TeX nical Institute\\[.75cm]
Certificate
\end{center}
\noindent This is to certify that Mr. N. O. Vice has undergone a
course at this institute and is qualified to be a \TeX nician.
\begin{flushright}
The Director\\
The \TeX nical Institute
\end{flushright}

\textit{A polygon of three sides is called a \emph{triangle} and a
polygon of four sides is called a \emph{quadrilateral}}

\begin{center}
{\bfseries\huge The \TeX nical Institute}\\[1cm]
{\scshape\LARGE Certificate}
\end{center}
\noindent This is to certify that Mr. N. O. Vice has undergone a
course at this institute and is qualified to be a \TeX nical Expert.
\begin{flushright}
{\sffamily The Director\\
The \TeX nical Institute}
\end{flushright}

\chapter{Chapter}
\section{Section}
\subsection{Subsection}
\subsubsection{Subsubsection}

\subsection{Example}
In this example, we show how subsections and subsubsections
are produced (there are no subsubsubsections). Note how the
subsections are numbered.
\subsubsection{Subexample}
Did you note that subsubsections are not numbered? This is so in the
\texttt{book} and \texttt{report} classes. In the \texttt{article}
class they too have numbers. (Can you figure out why?)
\paragraph{Note}
Paragraphs and subparagraphs do not have numbers. And they have
\textit{run-in} headings.
Though named ‘‘paragraph’’ we can have several paragraphs of text within this.
\subparagraph{Subnote}
Subparagraphs have an additional indentation too.
And they can also contain more than one paragraph of text.

\begin{enumerate}
\item prepare a source file with the extension ``tex''
\item Compile it with \LaTeX to produce a ``dvi'' file
\item Print the document using a ``dvi'' driver
\end{enumerate}

Let’s take stock of what we’ve learnt
\begin{tabbing}
\hspace{1cm}\= \textbf{AbiWord}\quad\= A word processor\\[5pt]
\> \textbf{Emacs} \> A text editor\\[5pt]
\> \textbf{\TeX} \> A typesetting program
\end{tabbing}

The table below shows the sizes of the planets of our solar system.
\begin{center}
\begin{tabular}{lr}
Planet & Diameter(km)\\[5pt]
Mercury & 4878\\
Venus & 12104\\
Earth & 12756\\
Mars & 6794\\
Jupiter & 142984\\
Saturn & 120536\\
Uranus & 51118\\
Neptune & 49532\\
Pluto & 2274
\end{tabular}
\end{center}
As can be seen, Pluto is the smallest and Jupiter the largest.

In the seventeenth century, Fermat conjectured that if $n>2$, then
there are no integers $x$, $y$, $z$ for which
$$
x^n+y^n=z^n.
$$
This was proved in 1994 by Andrew Wiles.

The sequence
$$
2\sqrt{2}\,,\quad 2^2\sqrt{2-\sqrt{2}}\,,\quad 2^3
\sqrt{2-\sqrt{2+\sqrt{2}}}\,,\quad 2^4\sqrt{2-
\sqrt{2+\sqrt{2+\sqrt{2+\sqrt{2}}}}}\,,\;\ldots
$$
converge to $\pi$.

In the classical \emph{syllogism}
\begin{enumerate}
\item All men are mortal.\label{pre1}
\item Socrates is a man.\label{pre2}
\item So Socrates is a mortal.\label{con}
\end{enumerate}
Statements (\ref{pre1}) and (\ref{pre2}) are the \emph{premises} and
statement (\ref{con}) is the conclusion.

\begin{minipage}{5cm}
Footnotes in a minipage are numbered
using lowercase letters.\footnote{%
Inside minipage} \par This text
references a footnote at the bottom
of the page.\footnotemark
\end{minipage}
\footnotetext{At bottom of page}

It is hard to write unstructured and disorganised documents using
\LaTeX˜\cite{les85}.It is interesting to typeset one
equation˜\cite[Sec 3.3]{les85} rather than setting ten pages of
running matter˜\cite{don89,rondon89}.
\begin{thebibliography}{9}
\bibitem{les85}Leslie Lamport, 1985. \emph{\LaTeX---A Document
Preparation System---User’s Guide and Reference Manual},
Addision-Wesley, Reading.
\bibitem{don89}Donald E. Knuth, 1989. \emph{Typesetting Concrete
Mathematics}, TUGBoat, 10(1):31-36.
\bibitem{rondon89}Ronald L. Graham, Donald E. Knuth, and Ore
Patashnik, 1989. \emph{Concrete Mathematics: A Foundation for
Computer Science}, Addison-Wesley, Reading.
\end{thebibliography}

\end{document}

